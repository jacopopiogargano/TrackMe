\documentclass[../../DD.tex]{subfiles}
\begin{document}
\section{Overview \label{sect:2.1}}
	%%%%%%%%%%%%%%%%%%%%%%%%%%%%%%%%%%%%%%% DATA4HELP OVERVIEW %%%%%%%%%%%%%%%%%%%%%%%%%%%%

	\image {13cm} {Overviews/OverviewData4Help.jpg} {Data4Help Overview} {OverviewData4Help}

	This diagram is a general overview of the system to be.\\
	The \ic{User} interacts with their smartphone performing several actions, including but not limited to managing their services, sending their consent to the sharing of their data with \ic{Third parties}, and using a \ic{Service}. The smartphone can communicate with Data4Help Server calling the offered REST APIs. Data4Help Server also establishes a socket connection with the smartphone so as to collect \ic{User data} at the maximum throughput frequency allowed by the \ic{User} \ic{Smart wearables}. The smartphone is also connected to the \ic{User's Smart wearable}, allowing for continuous data collection.\\
	Data4Help Server is the core of the system to be. It manages all \ic{User} requests, collects their data and sends them to the database - Data4Help DB Server. It also sends through a socket connection newly collected \ic{User data} only to those \ic{Services Users} are subscribed to. It allows the forwarding of \ic{Third Party User data} requests to \ic{Users} and supports the interaction of \ic{Third Parties} with the system through their dedicated website.
	Lastly, Third Party Service System communicates with the \ic{User's} smartphone via HTTP since Data4Help offers the possibility of embedding \ic{Third Party Service} websites into the Data4Help application.
	\image {13cm} {Overviews/OverviewAutomatedSOS.jpg} {AutomatedSOS Overview} {OverviewAutomatedSOS}
	AutomatedSOS is based on a central server which receives \ic{User data} as soon as it is produced and constantly analyzes it. The data is sent from Data4Help to AutomatedSOS via a socket connection, which is always active. When AutomatedSOS recognizes a \ic{User} as a \ic{User in need}, it calls Local Emergency Services and stores in the AutomatedSOS database all the relevant information, including which Local Emergency Services are in charge of assisting a specific \ic{User in need}.
	\image {13cm} {Overviews/OverviewTrack4Run.jpg} {Track4Run Overview} {OverviewTrack4Run}
	The overview of Track4Run gives a general idea of the several components that comprise this \ic{Service}.\\
	Track4Run Server manages all the actions that can be performed by the actors interacting with the \ic{Service}. First of all, it allows \ic{Organizers} to create a \ic{Run} and manage it through a dedicated website. Moreover, it communicates with Data4Help to allow \ic{Users} to enroll in a \ic{Run} as \ic{Participants} and acquire their real time data while running. This data will be displayed in the dedicated \ic{Spectators} website, together with the real time position of \ic{Participants}, retrieving the real world map of the \ic{Run} through Google Maps. All data that needs to be stored, is passed to and retrieved from a database - Track4Run DB Server. This data is mainly made up of information about \ic{Runs}.

	\end{document}
