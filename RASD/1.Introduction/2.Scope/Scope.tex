\documentclass[../../rasd.tex]{subfiles}
\begin{document}

\section{Scope}
			TrackMe offers its services in a world where technology and health are taking huge strides forward every day and innovation is commonplace.\\
Nowadays, people use smart devices such as smartphones and smart wearables more than any other object that they own. This means that any activity they perform already is or can be integrated with these devices.\\
TrackMe, with the introduction of Data4Help, offers the possibility to monitor users’ location and health data and allows third parties to register in the system to acquire these data.\\

When it comes to personal data acquisition, privacy is a fundamental issue that TrackMe needs to consider. Privacy is, in fact, regulated by several laws: there are many restrictions on how user’s data is acquired and stored. Therefore, TrackMe is concerned with users’ consent to transferring data to TrackMe itself and to third parties for individual specific analysis. Moreover, TrackMe guarantees that anonymized data of groups of individuals are properly anonymized by checking specific constraints.\\

Over the course of their daily routine, users perform several actions during which their data can be analyzed to provide them with insights. For instance, they might want to monitor their heart rate while sleeping or to keep track of the distance they have walked during their day and the places they have been to.\\

People with a potential need for immediate assistance have always been a huge concern for their relatives and for technology makers. These may include old people with limited movement and a high chance to need urgent assistance, anyone who has a specific disease, but also a healthy individual who can suffer from a sudden heart failure. Until now, the only practical way to receive help has been to call for help, either by using a cell phone or by pushing an SOS button on a dedicated device. TrackMe proposes to automatize the step of calling for help through AutomatedSOS. In fact, when determined health values will no more be considered as normal, the system will automatically send a request for help.\\

Another part of the market that TrackMe decided to enter, is the sport one, where having the possibility of collecting and sharing data about people is nowadays disruptive.\\
A sport practiced and loved by many is running. Organizing a run requires several steps to be taken such as defining a path, getting athletes to participate and spectators to watch it. TrackMe proposes to simplify the organization of a run, by introducing Track4Run. This service will allow the definition of a path, easy enrollment for participants and a real-time tracking of each runner’s position on a map.
			\subsection{Analysis of shared phenomena}
			TO DO LIST OF SHARED PHENOMENA
			
		\begin{enumerate}
		\item users move (or run in Track4Run)
		\item users can have health problems
		\item sensors collect data
		\item sensors communication
		\item sensors break
		\item third parties collect data from the system
		\item third parties registration to Data4Help
		\item user grant direct usage of personal data
		\item user registration (Data4Help and/or services built on top of it)
		\item organizers of run define path
		\item participants of run enroll to it
		\item run spectators see on a map the position of runners
		
		\end{enumerate}

\end{document}