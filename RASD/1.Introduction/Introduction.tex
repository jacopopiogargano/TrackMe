\providecommand{\rasd}{..}
\documentclass[../rasd.tex]{subfiles}

\begin{document}

\chapter{Introduction}
\thispagestyle{fancy}
		
		\subfile{\rasd/1.Introduction/1.Purpose/Purpose.tex}
		\subfile{\rasd/1.Introduction/2.Scope/Scope.tex}
		\subfile{\rasd/1.Introduction/3.Goals/Goals.tex}
		\subfile{\rasd/1.Introduction/4.DefinitionsAcronymsAbbreviations/DefinitionsAcronymsAbbreviations.tex}
	
		\section{Revision History}
		\begin{enumerate}
			\item Version 1.0 - 11\sups{th} November 2018
		\end{enumerate}
		\section{Reference Documents}
			\begin{itemize}
				\item Rumbaugh, Jacobson, Booch. 1999. \ic{The Unified Modeling Language Reference Manual}. Addison-Wesley.
				\item MIT Software Design Group. \ic{Appendix B: Alloy Language Reference}. alloytools.org/documentation.html
			\end{itemize}
		\section{Document Structure}
		This document is divided into six main chapter.
		\paragraph{First chapter}
		It introduces the project in terms of purpose, scope and goals. Moreover, it contains the definitions, acronyms and abbreviations needed to properly understand the following sections. All the documents used to write this one are all listed to enable a fast references check.
		\paragraph{Second chapter}
		It goes deep into the system description and definition. In particular it describes:
		\begin{itemize}
			\item Detailed description of the product to be delivered and its features
			\item List of all domain assumptions that enables to reach the goals described in chapter 1
			\item Description of users for which is thought the product
		\end{itemize}
		\paragraph{Third chapter}
		It analyze and lists all the requirements needed to reach the goals described in chapter 1. Requirements is a huge and fundamental part of the project lyfe cycle so the chapter is divided in many section, once for every aspect of requirements definition.
		\begin{itemize}
			\item Functional Requirements listed as description of interfaces, scenarios and uses cases (use cases and sequence diagrams are included)
			\item Non Functional Requirements (performance, reliability, availability, security, maintainability, portability)
			\item Any design constraints
		\end{itemize}
		\paragraph{Fourth chapter}
		It is dedicated to the formalization of the system and its scope through a formal representation of entities and constraints using Alloy.
		\paragraph{Fifth chapter}
		It is just intended for statistical reasons: effort spent by all team members is shown as the list of all activities done during the realization of this document.
		\paragraph{Sixth chapter}
		\todo{TODO...}
		
\end{document}