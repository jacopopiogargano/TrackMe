\documentclass{report}
\usepackage{xcolor,comment, subfiles, graphicx, caption, longtable, subfig, fancyhdr}
\usepackage{float} %Used to force image to appear in the section in which it's declared
\usepackage[hidelinks]{hyperref}
% SECTION COMMANDS: here we include the commands for formatting and other purposes
\newcommand\todo[1]{\textcolor{red}{#1}}

%\renewcommand{\headrulewidth}{0pt} % erases the rule under the header
%\renewcommand{\footrulewidth}{0.3pt} % erases the rule under the header


\newcommand{\sups}[1]{\ensuremath{^{\textrm{#1}}}}

\newcommand{\subs}[1]{\ensuremath{_{\textrm{#1}}}}

\newcommand{\ic}[1]{\textit{#1}}

\newcommand{\image}[4]{
	\begin{figure}[H]
	\centering
	\includegraphics[width=\linewidth, height = {#1}, keepaspectratio]{#2}
	\caption*{#3}
	\label{#4}

	\label{fig:nonfloat} %Used to force image to appear in the section in which it's declared
	\end{figure}
}

%Alligns two images in a row 
\newcommand{\twoimages}[5]{
	\captionsetup[subfigure]{labelformat = empty}
	\begin{figure}[H]
		\begin{center}
	        	\subfloat[#3]{
			\includegraphics[height=#1, keepaspectratio]{#2}
	        	}
        		\hspace{25mm}
	        	\subfloat[#5]{
			\includegraphics[height=#1, keepaspectratio]{#4}
		}
		\end{center}
	\end{figure}
}

\providecommand{\rasd}{.} % relative path to rasd.tex

\graphicspath{{Images/}}



%\title{
%\small Politecnico di Milano\\
%\small AA 2018-2019\\
%\large Computer Science and Engineering\\
%\Large Software Engineering 2 Project\\
%\huge Requirement Analysis and Specification Document
%}




%\date{2018-11-11}
%\author{
%\large Gargano Jacopo Pio, Giannetti Cristian, Haag Federico}

\pagestyle{fancy}
\fancyhf{}
\rhead{\color{gray}{\normalsize{TrackMe - RASD - 1.0}}}
\lfoot{\textcolor{gray}{\small{Copyright © 2018, Gargano Jacopo Pio, \newline Giannetti Cristian, Haag Federico – All rights reserved}}}
\rfoot{\textcolor{gray}{\newline \thepage}}
\addtocontents{toc}{\protect\thispagestyle{fancy}}


\begin{document}
	\pagenumbering{gobble}
	\begin{titlepage}
		\centering	
		\vfill
		{
			\includegraphics[width =\linewidth, height = 4cm, keepaspectratio]{PolitecnicoLogo.png}
			\label{fig:PolitecnicoLogo}
			\large \\[2ex]M.Sc. Computer Science and Engineering\\
			\large Software Engineering 2 Project\\[9ex]			
			\image{5cm}{TrackMeLogo.png}{}{TrackMeLogo}
			\huge Requirement Analysis and Specification Document\\[4ex]

			\normalsize Gargano Jacopo Pio, Giannetti Cristian, Haag Federico\\[1.5ex]
			\normalsize 11 November 2018 \\[1.5ex]
			\normalsize GitHub Repository: https://github.com/federicohaag/GarganoGiannettiHaag\\[3ex]
			\normalsize Version 1.0


		}
		
	\end{titlepage}
	%\maketitle

	
	\newpage
	\pagenumbering{arabic}
	\tableofcontents
	\thispagestyle{fancy}
%	\let\tableofcontents\relax
	
	\newpage
	
	\subfile{1.Introduction/Introduction.tex}
	\subfile{2.OverallDescription/OverallDescription.tex}
	\subfile{3.SpecificRequirements/SpecificRequirements.tex}
	\subfile{4.FormalAnalysisUsingAlloy/FormalAnalysisUsingAlloy.tex}
		
	\chapter{Effort Spent}
	\thispagestyle{fancy}
		\paragraph{Gargano Jacopo Pio} Total hours of work: 64h
			\begin{itemize}
				\item 2h Reading of Project Delivery Document, General LaTex setting.
				\item 1h Scope
				\item 3h RASD Review homework
				\item 3h Product functions - Data4Help
				\item 1h Creating subfiles structure in Latex
				\item 4h Functional Requirements and Goals definition 
				\item 4h Class diagrams, Goals redefinition and revision, Document Structure fix
				\item 5h Use Cases
				\item 3h Scenarios
				\item 2h General revision
				\item 1h State Charts
				\item 3h Requirements
				\item 3h Satisfying Goals
				\item 3h Alloy, General revision
				\item 3h Add Constraints, Shared phenomena review
				\item 2h Class Diagrams
				\item 3h General revision, Performance Requirements
				\item 1h Meeting with Professor
				\item 5h Alloy
				\item 10h Alloy
				\item 2h General Revision

			\end{itemize}
		\paragraph{Giannetti Cristian} Total hours of work: 62.5h
			\begin{itemize}
				\item 2h Reading of Project Delivery Document, General LaTex setting.
				\item 1h Goals
				\item 3h RASD Review homework
				\item 1h Assumptions
				\item 3h Functional Requirements, Goals 
				\item 2h Class diagrams
				\item 3h Goals revision, Domain Assumptions
				\item 3h Use Cases
				\item 1h State charts
				\item 3h Mockup
				\item 0.5h Availability, Portability
				\item 3h State charts
				\item 1h Requirements
				\item 1h Use cases diagrams
				\item 3h Sequence diagrams
				\item 5h Mockup
				\item 1h Sequence Diagrams
				\item 2h Mockup
				\item 4h Mockup
				\item 3h Sequence Diagrams, Use Case Diagrams
				\item 2h General Revision
				\item 3h Diagrams Revision
				\item 10h Alloy
				\item 2h General Revision

			\end{itemize}
		\paragraph{Haag Federico} Total hours of work: 47h
			\begin{itemize}
				\item 2h Reading of Project Delivery Document, General LaTex setting.
				\item 0.5h Purpose
				\item 3h RASD Review homework
				\item 1h User characteristics
				\item 4h Functional Requirements, Goals
				\item 1h Class diagrams
				\item 3h General revision,Product perspective
				\item 1h Revision of domain assumptions, Sse cases
				\item 1h Software Interfaces
				\item 0.5h Hardware Interfaces
				\item 6h General revision, Use Case diagrams, Sequence diagrams, Requirements
				\item 2h Revision of sequence diagram, Alloy
				\item 1h Meeting with Professor
				\item 1h Revision of Requirements
				\item 1h Revision of Class diagrams
				\item 3h Alloy
				\item 2h General revision, Effort Spent, Communication Interfaces, Fix of Definitions
				\item 2h Document Structure, Software System Attributes
				\item 10h Alloy
				\item 2h General Revision
				
			\end{itemize}
	\chapter{References}
	\thispagestyle{fancy}
	\begin{itemize}
				\item[1]E. Di Nitto. \ic{Lecture Slides}. Politecnico di Milano.
				\item[2]E. Di Nitto. \ic{Mandatory Project Assignment AY 2018-2019}. Politecnico di Milano.
				\item[3]ISO/IEC/IEEE 29148:2011. \ic{Standard on requirement engineering}.\\https://standards.ieee.org/standard/29148-2011.html.
	\end{itemize}
	
\end{document}
