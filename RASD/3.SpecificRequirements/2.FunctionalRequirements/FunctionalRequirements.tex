\documentclass[../../rasd.tex]{subfiles}
\begin{document}

\section{Functional Requirements}
		%requirements and doamin assumptions entail goals


		\subsection{Scenarios}
            \paragraph{Data4Help}
                \begin{itemize}
                    
                    %user registration and log in                
                    \item[S\subs{1}] Dante is an individual who would like to keep track of his GPS position and health data. For this purpose he decides to use Data4Help. He downloads the Data4Help application on his smartphone and proceeds to sign up. He inserts all required information, which include his name, his social security number and date of birth. He is asked to insert an email that will later be his username and a password. Dante inserts his name as his password and the system tells him that the inserted password is shorter than 8 characters, so he tries again with a new one. Eventually he inserts a valid password, accepts the terms and conditions and taps on "Create an Account". He is successfully signed up, after receiving a confirmation email by TrackMe. He tries to log into the application by inserting the newly created username and password. The system accepts the credentials and Dante is in.
                    
                    %Third Party registration and log in                
                    \item[S\subs{2}] YourHealth is a company that analyzes individuals' health data to provide users with insights on their well-being. It decides to offer its \ic{Service} also on Data4Help so as to have a greater pool of users. The person in charge navigates to the \ic{Third Party} dedicated website and clicks on "Sign Up". They fill in all required information about their company, insert an email that will be used as username and a valid password. They then accept the terms and conditions by clicking the specific checkbox, and finally click on "Create an Account". YourHealth receives an email confirming the account creation: now YourHealth \ic{Services} are available to all Data4Help \ic{Users}.
                    
                    %User adds a service and consequently Third Party requests user data
                    \item[S\subs{3}] Dante, a Data4Help \ic{User}, needs to monitor his heart rate through the day. He navigates to the "Discover" page inside of his Data4Help application and scrolls through the available \ic{Services}. He finds My\ding{170}3, a \ic{Service} developed by YourHealth, a \ic{Third Party} registered to Data4Help. The description of the service seems to suit his need, so he adds My\ding{170}3 to his \ic{Services}. In order to finalize the subscription, Dante will have to accept that his data will be sent to YourHealth for analysis. He does so. After a while, in the specific My\ding{170}3 \ic{Service} page, the "Analyze" button appears. Dante taps on it and promptly he sees a personalized graph showing his heart rate levels throughout the day, starting from the first day he registered to Data4Help.

                    %Third Party requests group data
                    \item[S\subs{4}] LocalStats is a company that performs intensive statistics on individuals' positions in some cities of Switzerland. It decides to acquire individuals GPS locations data from Data4Help to enlarge its database. LocalStats registers as a Data4Help \ic{Third Party}. Once registration is complete, the first request it makes to Data4Help refers to all female \ic{Users} between 30 and 35 years old living in Lausanne. Unfortunately, the number of \ic{Users} with the requested characteristics is less than 1000, which does not guarantee proper data anonymization. Therefore, Data4Help rejects the \ic{Group Data} request. LocalStats tries again changing the interval of interest to 25-35 years old. This time the request refers to more than 1000 \ic{Users} and finally Data4Help can send the requested \ic{Group Data} to LocalStats.

                    %Third Party subscribes to new data (forse da includere in uno dei due precedenti)
                    \item[S\subs{5}] Dante, from scenario S\subs{3}, would like to keep My\ding{170}3 active day by day. To do so, he taps on "Analyze Daily", which is a function offered by My\ding{170}3. YourHealth, which developed and manages My\ding{170}3, requests subscription to Dante's new data. Data4Help registers that anytime Dante's \ic{User data} is collected, it needs to send it to YourHealth for analysis. Starting from the following day, Dante does not need anymore to tap on "Analyze" every day: new analysis is provided to him as soon as it is available from My\ding{170}.

                    %user revokes consent to a Third Party service
                    \item[S\subs{6}] Dante, who subscribed to Data4Help and used its \ic{Third Party Services} for a while, decides that he does not want to use one of them, TrackKer, anymore. Therefore, he navigates to the "My Services" page and taps on TrackKer. The \ic{Service} page shows up and he taps on the "Revoke Consent" button at the bottom of the page. From now on, Data4Help will stop sending Dante's data to the \ic{Third Party} managing TrackKer.
                \end{itemize}

            \paragraph{AutomatedSOS}
                \begin{itemize}
                    %register anomalous data for a user, collect location and call ambulance                
                    \item[S\subs{7}] GianVito is a 57 years old man subscribed to AutomatedSOS. After getting very angry at work, he drives home, but as soon as he gets there he feels dizzy and falls on the ground. He is alone and cannot call for help. Fortunately, AutomatedSOS notices that his heart rate is below a certain threshold and identifies him as \ic{User in need}. AutomatedSOS calls the local emergency services and sends them GianVito's position and health data. When the local emergency services dispatch an ambulance and GianVito is being taken care of, AutomatedSOS finished handling his needs and starts monitoring his health data again.
                \end{itemize}

            \paragraph{Track4Run}
                \begin{itemize}
                    %organize run                
                    \item[S\subs{8}] Charity4All is a Swedish charity association that organizes a running competition every year to raise money for their causes. The person in charge decides to use Track4Run to manage the run. They navigate to the \ic{Run} dedicated website and sign up as an \ic{Organizer}, inserting an email and a password for registration. Once sign up is complete, they click on "Create Run" and the \ic{Run} creation page shows up. They give the \ic{Run} a name - Run4Char - they define a path around Gothenburg and set the date and time the competition will take place on. They do not want to limit the number of participants, so they click on "Create Run" and obtain a \ic{Run} identifier back from Track4Run. They will distribute this identifier to all viewers who wish to enjoy the \ic{Run} on their devices.

                    %participant enrolls in run
                    \item[S\subs{9}] Hannah lives in Gothenburg and she loves running. In fact, she is subscribed to Track4Run. While browsing the available \ic{Runs} in her city, she finds Run4Char from S\subs{8} \todo{add ref}. She enrolls in the run right away and Track4Me records her registration. Hannah is now a \ic{Participant} of the \ic{Run}.  

                    %spectator wants to watch run
                    \item[S\subs{10}] George enjoys sports a lot, however he is very old now and cannot participate in competitions anymore. He still likes watching sports event, especially when it comes to running. Since he is also into helping others, he is subscribed to Charity4All (from S\subs{8} \todo{add ref}) newsletter. He reads that they are organizing a \ic{Run} and writes down the \ic{Run} identifier. On the day of the \ic{Run}, he navigates to the \ic{Spectators} dedicated website and inserts the \ic{Run} identifier. As soon as the \ic{Run} starts, he enjoys it by watching the position of the \ic{Participants} on the map right on his device, comfortably in his house.
                \end{itemize}




		\subsection{Use Cases}

			\paragraph{Data4Help}
			\begin{center}
    			\begin{longtable}{| p{.35\linewidth} | p{.65\linewidth} |}
    			\hline
   				Name & User Sign Up\\ \hline
    			Actors & \ic{User}, Data4Help\\ \hline
    			Entry Conditions & \ic{User} successfully installed Data4Help application on their smartphone.\\ \hline
    			Events Flow & 
    				\begin{enumerate}
    					\item \ic{User} taps on "Sign Up" button.
    					\item \ic{User} fills in all required fields for \ic{User} registration, including username and password.
                        \item \ic{User} checks the "Accept terms and conditions" checkbox.
    					\item \ic{User} taps on "Create an Account" button.
    					\item Data4Help saves \ic{User} information.
    				\end{enumerate}
    			 \\ \hline
    			Exit Condition & \ic{User} successfully registered by Data4Help.\\ \hline
    			Exceptions & 
    			\begin{enumerate}
    					\item Inserted email already registered for another \ic{User}.
    					\item Inserted password is not valid.
    					\item Not all required fields are filled in.
                        \item "Accept terms and conditions" checkbox not checked.
    					\item \ic{User} already signed up.
    				\end{enumerate}
    				\ic{User} is invited to try again signing up, reporting which error(s) they have committed.
    				 \\ \hline
    				
    				
       			\end{longtable}
			\end{center}

			\begin{center}
    			\begin{longtable}{| p{.35\linewidth} | p{.65\linewidth} |}
    			\hline
   				Name & Third Party Sign Up\\ \hline
    			Actors & \ic{Third Party}, Data4Help \\ \hline
    			Entry Conditions & \ic{Third Party} is connected to the \ic{Third Party} dedicated website (see \todo{add reference}).\\ \hline
    			Events Flow & 
    				\begin{enumerate}
    					\item \ic{Third Party} clicks on "Sign Up" button.
    					\item \ic{Third Party} fills in all required fields for \ic{Third Party} registration, including username and password.
                        \item \ic{Third Party} checks the "Accept terms and conditions" checkbox.
    					\item \ic{Third Party} clicks on "Create an Account" button.
    					\item Data4Help saves \ic{Third Party} information.
    				\end{enumerate}
    			 \\ \hline
    			Exit Condition & \ic{Third Party} successfully registered by Data4Help.\\ \hline
    			Exceptions & 
    			\begin{enumerate}
    					\item Inserted email already registered for another \ic{Third Party}.
    					\item Inserted password is not valid.
    					\item Not all required fields are filled in.
                        \item "Accept terms and conditions" checkbox not checked.
    					\item \ic{Third Party} already signed up.
    				\end{enumerate}
    				\ic{Third Party} is invited to try again signing up, reporting which error(s) it has committed.
    				 \\ \hline
    			\end{longtable}
			\end{center}

			\begin{center}
    			\begin{longtable}{| p{.35\linewidth} | p{.65\linewidth} |}
    			\hline
   				Name & User Log In\\ \hline
    			Actors & \ic{User}, Data4Help \\ \hline
    			Entry Conditions & \ic{User} successfully registered to Data4Help and installed Data4Help application on their smartphone.\\ \hline
    			Events Flow & 
    				\begin{enumerate}
    					\item \ic{User} enters username.
    					\item \ic{User} enters password.
    					\item \ic{User} taps on "Log In" button.
    					\item Data4Help checks \ic{User} credentials.
    				\end{enumerate}
    			 \\ \hline
    			Exit Condition & \ic{User} is successfully logged in.\\ \hline
    			Exceptions & 
    			\begin{enumerate}
    					\item Inserted username is not valid.
    					\item Inserted password is not correct.
    			\end{enumerate}
    			\ic{User} is invited to try again logging in.
    				 \\ \hline
    			\end{longtable}
			\end{center}

			\begin{center}
    			\begin{longtable}{| p{.35\linewidth} | p{.65\linewidth} |}
    			\hline
   				Name & Third Party Log In\\ \hline
    			Actors & \ic{Third Party}, Data4Help \\ \hline
    			Entry Conditions & \ic{Third Party} successfully registered to Data4Help and is connected to the \ic{Third Party} dedicated website (see \todo{add reference}).\\ \hline
    			Events Flow & 
    				\begin{enumerate}
    					\item \ic{Third Party} enters username.
    					\item \ic{Third Party} enters password.
    					\item \ic{Third Party} clicks on "Log In" button.
    					\item Data4Help checks \ic{Third Party} credentials.
    				\end{enumerate}
    			 \\ \hline
    			Exit Condition & \ic{Third Party} is successfully logged in.\\ \hline
    			Exceptions & 
    			\begin{enumerate}
    					\item Inserted username is not valid.
    					\item Inserted password is not correct.
    			\end{enumerate}
    			\ic{Third Party} is invited to try again logging in .
    				 \\ \hline
    			\end{longtable}
			\end{center}

			\begin{center}
    			\begin{longtable}{| p{.35\linewidth} | p{.65\linewidth} |}
    			\hline
   				Name & Third Party requests User Data\\ \hline
    			Actors & \ic{Third Party}, Data4Help, \ic{User} \\ \hline
    			Entry Conditions & \ic{Third Party} and \ic{User} successfully registered to Data4Help.\\ \hline
    			Events Flow & 
    				\begin{enumerate}
    					\item \ic{Third Party} requests access to specific \ic{User data}.
    					\item Data4Help forwards the request to the specific \ic{User} unless the consent was already given.
    					\item \ic{User} gives consent to the requesting \ic{Third Party} to access their data.
    				\end{enumerate}
    			 \\ \hline
    			Exit Condition & Data4Help sends \ic{User data} to the \ic{Third Party}.\\ \hline
    			Exceptions & 
    			\begin{enumerate}
    					\item \ic{User} denies consent to their data access by the requesting \ic{Third Party}.
    			\end{enumerate}
    				 \\ \hline
    			\end{longtable}
			\end{center}

			\begin{center}
    			\begin{longtable}{| p{.35\linewidth} | p{.65\linewidth} |}
    			\hline
   				Name & Third Party requests Group Data\\ \hline
    			Actors & \ic{Third Party}, Data4Help \\ \hline
    			Entry Conditions & \ic{Third Party} successfully registered to Data4Help.\\ \hline
    			Events Flow & 
    				\begin{enumerate}
    					\item \ic{Third Party} requests access to \ic{Group data}.
    					\item Data4Help checks if the requested data refers to minimum 1000 \ic{Users}.
    				\end{enumerate}
    			 \\ \hline
    			Exit Condition & Data4Help sends \ic{Group data} to the \ic{Third Party}.\\ \hline
    			Exceptions & 
    			\begin{enumerate}
    					\item \ic{Group data} refers to less than 1000 \ic{Users}.
    			\end{enumerate}
    				 Data4Help denies \ic{Group data} access to the \ic{Third Party}.
    				 \\ \hline
    			\end{longtable}
			\end{center}

            \begin{center}
                \begin{longtable}{| p{.35\linewidth} | p{.65\linewidth} |}
                \hline
                Name & Third Party subscribes to New User Data\\ \hline
                Actors & \ic{Third Party}, Data4Help \\ \hline
                Entry Conditions & \ic{Third Party} successfully registered to Data4Help and obtained access to \ic{User data}.\\ \hline
                Events Flow & 
                    \begin{enumerate}
                        \item \ic{Third Party} requests subscription to \ic{User data}.
                        \item \ic{User} gives consent.
                    \end{enumerate}
                 \\ \hline
                Exit Condition & Data4Help registers the \ic{Third Party} subscription to new data. Each time new data is produced, it is sent to the \ic{Third Party}.\\ \hline
                Exceptions & No Exceptions
                     \\ \hline
                \end{longtable}
            \end{center}
            
            \begin{center}
                \begin{longtable}{| p{.35\linewidth} | p{.65\linewidth} |}
                \hline
                Name & Third Party subscribes to New Group Data\\ \hline
                Actors & \ic{Third Party}, Data4Help \\ \hline
                Entry Conditions & \ic{Third Party} successfully registered to Data4Help and obtained access to \ic{Group data}.\\ \hline
                Events Flow & 
                    \begin{enumerate}
                        \item \ic{Third Party} requests subscription to \ic{Group data}.
                    \end{enumerate}
                 \\ \hline
                Exit Condition & Data4Help registers the \ic{Third Party} subscription to new data. Each time new data is produced, it is sent to the \ic{Third Party}.\\ \hline
                Exceptions & 
                No exceptions
                     \\ \hline
                \end{longtable}
            \end{center}

            \begin{center}
                \begin{longtable}{| p{.35\linewidth} | p{.65\linewidth} |}
                \hline
                Name & User subscribes to a Service\\ \hline
                Actors & \ic{Third Party}, \ic{User}, Data4Help \\ \hline
                Entry Conditions & \ic{User} is successfully registered to Data4Help and installed Data4Help application on their smartphone.\\ \hline
                Events Flow & 
                    \begin{enumerate}
                        \item \ic{User} navigates to "Discover" page.
                        \item \ic{User} chooses which \ic{Service} they would like to subscribe to.
                        \item \ic{User} taps on "Add" button.
                        \item \ic{User} gives consent to sharing their data with the specific \ic{Third Party}.
                    \end{enumerate}
                 \\ \hline
                Exit Condition & Data4Help registers the new \ic{Service} for the \ic{User}\\ \hline
                Exceptions & 
                \begin{enumerate}
                        \item \ic{User} does not give consent to sharing their data.
                \end{enumerate}
                     The \ic{Service} is not added and the \ic{User} is invited to try again adding it.
                     \\ \hline
                \end{longtable}
            \end{center}

            \begin{center}
                \begin{longtable}{| p{.35\linewidth} | p{.65\linewidth} |}
                \hline
                Name & User revokes consent to Third Party Service\\ \hline
                Actors & \ic{Third Party}, \ic{User}, Data4Help \\ \hline
                Entry Conditions & \ic{User} gave consent to sharing their data with a \ic{Third Party}.\\ \hline
                Events Flow & 
                    \begin{enumerate}
                        \item \ic{User} navigates to "My Services" page.
                        \item \ic{User} chooses which service they would like to revoke consent.
                        \item \ic{User} navigates to the \ic{Service} dedicated page by tapping on its name.
                        \item \ic{User} taps on "Revoke consent" button.
                    \end{enumerate}
                 \\ \hline
                Exit Condition & Data4Help stops sharing the data of the \ic{User} with the specific \ic{Third Party}.\\ \hline
                Exceptions & 
                \begin{enumerate}
                        \item \todo{???}
                \end{enumerate}
                     \todo{???}
                     \\ \hline
                \end{longtable}
            \end{center}

			
            \todo{? ? ? Collect data --> probably it is not a use case, but regular product function
			? ? ? User manages data (gdpr)}


			\paragraph{AutomatedSOS}

            \begin{center}
                \begin{longtable}{| p{.35\linewidth} | p{.65\linewidth} |}
                \hline
                Name & User in need assisted by AutomatedSOS\\ \hline
                Actors & \ic{User}, AutomatedSOS, Local emergency services \\ \hline
                Entry Conditions & \ic{User} is subscribed to AutomatedSOS and installed Data4Help application on their smartphone.\\ \hline
                Events Flow & 
                    \begin{enumerate}
                        \item AutomatedSOS identifies {User data} as \ic{anomalous data}.
                        \item \ic{User} is identified as \ic{User in need}.
                        \item AutomatedSOS calls local emergency services requesting an ambulance.
                        \item AutomatedSOS sends \ic{User data} including GPS location and health data to local emergency services.
                        \item Local emergency services send an ambulance to the location of the \ic{User in need}.
                    \end{enumerate}
                 \\ \hline
                Exit Condition & \ic{User in need} is assisted by local emergency services.\\ \hline
                Exceptions & 
                \begin{enumerate}
                        \item Local emergency services don't answer the call
                        \item The call to local emergency services is answered but the communication fails before giving all the details
                \end{enumerate}
                     The call is repeated.
                     \\ \hline
                \end{longtable}
            \end{center}

        
			\paragraph{Track4Run}
			
			\begin{center}
    			\begin{longtable}{| p{.35\linewidth} | p{.65\linewidth} |}
    			\hline
   				Name & Organizer Sign Up\\ \hline
    			Actors & \ic{Organizer}, Track4Run\\ \hline
    			Entry Conditions & \ic{Organizer} is connected to the Track4Run website dedicated to organizers.\\ \hline
    			Events Flow & 
    				\begin{enumerate}
    					\item \ic{Organizer} taps on "Sign Up" button.
    					\item \ic{Organizer} fills in all required fields for \ic{Organizer} registration, including username and password.
                        \item \ic{Organizer} checks the "Accept terms and conditions" checkbox.
    					\item \ic{Organizer} taps on "Create an Account" button.
    					\item Track4Run saves \ic{Organizer} information.
    				\end{enumerate}
    			 \\ \hline
    			Exit Condition & \ic{Organizer} successfully registered by Track4Run.\\ \hline
    			Exceptions & 
    			\begin{enumerate}
    					\item Inserted email already registered for another \ic{Organizer}.
    					\item Inserted password is not valid.
    					\item Not all required fields are filled in.
                        \item "Accept terms and conditions" checkbox not checked.
    					\item \ic{Organizer} already signed up.
    				\end{enumerate}
    				\ic{Organizer} is invited to try again signing up, reporting which error(s) they have committed.
    				 \\ \hline
    				
    				
       			\end{longtable}
			\end{center}
			
            \begin{center}
                \begin{longtable}{| p{.35\linewidth} | p{.65\linewidth} |}
                \hline
                Name & Organizer creates a Run\\ \hline
                Actors & \ic{Organizer}, Track4Run \\ \hline
                Entry Conditions & \ic{Organizer} is connected to the \ic{Run} dedicated website (see \todo{add reference}).\\ \hline
                Events Flow & 
                    \begin{enumerate}
                        \item \ic{Organizer} clicks on "Create Run" button.
                        \item \ic{Organizer} fills in all required fields for \ic{Run} creation.
                        \item \ic{Organizer} clicks on ''Confirm'' button.
                    \end{enumerate}
                 \\ \hline
                Exit Condition & Track4Run creates the \ic{Run} defined by the \ic{Organizer}.\\ \hline
                Exceptions & 
                \begin{enumerate}
                        \item Not all required fields are filled in.
                \end{enumerate}
                     \ic{Organizer} is invited to try again creating the \ic{Run}.
                     \\ \hline
                \end{longtable}
            \end{center}

            \begin{center}
                \begin{longtable}{| p{.35\linewidth} | p{.65\linewidth} |}
                \hline
                Name & User enrolls in a Run\\ \hline
                Actors & \ic{User}, Track4Run \\ \hline
                Entry Conditions & \ic{User} is subscribed to Track4Run and installed Data4Help application on their smartphone.\\ \hline
                Events Flow & 
                    \begin{enumerate}
                        \item \ic{User} navigates to "My Services" page.
                        \item \ic{User} taps on Track4Run.
                        \item \ic{User} taps on the \ic{Run} they wish to enroll in or inserts the run identifier.
                        \item \ic{User} taps on "Enroll" button.
                    \end{enumerate}
                 \\ \hline
                Exit Condition & Track4Run registers the \ic{User} as a \ic{Participant} of the \ic{Run}.\\ \hline
                Exceptions & 
                \begin{enumerate}
                        \item No \ic{Runs} are listed.
                        \item The selected \ic{Run} reached the maximum number of participants.
                        \item The selected \ic{Run} is closed to enrollment.
                        \item There are no \ic{Runs} associated to the inserted run identifier
                \end{enumerate}
                     \ic{Participant} is invited to try again later enrolling in a \ic{Run}.
                     \\ \hline
                \end{longtable}
            \end{center}

            \begin{center}
                \begin{longtable}{| p{.35\linewidth} | p{.65\linewidth} |}
                \hline
                Name & Spectator watches a Run\\ \hline
                Actors & Individual \\ \hline
                Entry Conditions & Individual is connected to the \ic{Spectators} dedicated website (see \todo{add reference}).\\ \hline
                Events Flow & 
                    \begin{enumerate}
                        \item \ic{Individual} clicks on the \ic{Run} they would like to watch or inserts the run identifier.
                        \item \ic{Individual} watches the \ic{Run} as a \ic{Spectator}. 
                    \end{enumerate}
                 \\ \hline
                Exit Condition & The \ic{Run} is over.\\ \hline
                Exceptions & 
                \begin{enumerate}
                        \item No \ic{Runs} are listed.
                        \item There is no \ic{Run} associated to the inserted run identifier
                        \item The \ic{Spectator} disconnects from the \ic{Spectators} dedicated website.
                \end{enumerate}
                     Concerning Exception 1 (\todo{add reference}), \ic{Spectator} is invited to try again later watching a \ic{Run}. For Exception 2 (\todo{add reference}), no action is taken.
                     \\ \hline
                \end{longtable}
            \end{center}

            \todo{??? participant runs in a run ???}

\end{document}