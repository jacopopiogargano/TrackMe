\documentclass[../../rasd.tex]{subfiles}
\begin{document}

\section{User characteristics}
\subsection{Data4Help}
\paragraph{Users:}
	People having at least a device with a sensor connected to the Internet, willing to share their data (see \ic{User data} in Section 1.4.1) with TrackMe so as to use the \ic{Services} built on top of Data4Help.
\paragraph{Third parties:}
	Companies or private persons willing to collect bulk data. This data is mainly used for building \ic{Services} on top of Data4Help; for many of these \ic{Services} it is very important that data is transferred in real time. Otherwise data may be used for statistical analysis. In both cases, \ic{Third parties} need that collected data is correct and accurate.

\subsection{AutomatedSOS}
\paragraph{Users:}
	People with a high probability of needing immediate assistance. AutomatedSOS users are willing to monitor their health parameters and GPS location to prevent finding themselves alone when in need. These are mainly elderly people, especially those living by themselves. However, all categories of people may want to use AutomatedSOS, specifically those suffering from a disease that may strike any moment.

\subsection{Track4Run}
\paragraph{Organizers:}
	Companies or private persons organizing \ic{Runs} willing to better engage \ic{Spectators} giving them the possibility to track in real-time the position of all \ic{Participants}. They need to provide this \ic{Service} easily in order to ensure \ic{Spectators} and \ic{Participants} are not prevented from using it.
\paragraph{Participants:}
	People participating in a \ic{Run}. They need to have a small device with no required interaction during the \ic{Run} so as to avoid distractions.

\paragraph{Spectators:}
	People participating as spectators of \ic{Runs}. They are willing to enjoy the event by tracking \ic{Participants} during all the \ic{Run}. Watching a \ic{Run} must be easy: no need of particular devices or installed applications.

\end{document}